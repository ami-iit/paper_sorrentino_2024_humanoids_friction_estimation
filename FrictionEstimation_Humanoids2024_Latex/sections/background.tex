\section{Background}
\label{sec:background}

\subsection{Notation}
\begin{itemize}
\item $I_{n}$ and $0_{m \times n}$ denote the $n \times n$ identity matrix and the $m \times n$ zero matrix respectively.
\item $\mathcal{I}$ denotes the inertial frame.
\item $\prescript{\mathcal{I}}{}{p}_\mathcal{B}$ is a vector connecting the origin of frame $\mathcal{I}$ and the origin of frame $\mathcal{B}$ expressed in frame $\mathcal{I}$.
\item Given $\prescript{\mathcal{I}}{}{p}_\mathcal{B}$ and $\prescript{\mathcal{B}}{}{p}_\mathcal{C}$,  $\prescript{\mathcal{I}}{}{p}_\mathcal{C} = \prescript{\mathcal{I}}{}{R}_\mathcal{B} \prescript{\mathcal{B}}{}{p}_\mathcal{C} + \prescript{\mathcal{I}}{}{p}_\mathcal{B}$.
% \item $\scalebox{0.8}{$\prescript{\mathcal{B}}{}{\textrm{v}}_\mathcal{I,B} = \begin{bmatrix} \prescript{\mathcal{B}}{}{v}_\mathcal{I,B} ^\top & \prescript{\mathcal{B}}{}{\omega}_\mathcal{I,B}^\top \end{bmatrix}^\top \in \mathbb{R}^6$}$ is the velocity of the frame $\mathcal{B}$ with respect to the frame $\mathcal{I}$ expressed in the frame $\mathcal{B}$, where $\prescript{\mathcal{B}}{}{v}_\mathcal{I,B}$ denotes the linear velocity and $\prescript{\mathcal{B}}{}{\omega}_\mathcal{I,B}$ the angular velocity. For the sake of simplicity, in the rest of the paper, we use ${\textrm{v}}_\mathcal{B}$.
\item $i_m \in \mathbb{R}^n$ is the vector of motor currents with $n$ the number of motors. 
\item $\scalebox{0.8}{$\prescript{}{}{\mathrm{f}}_\mathcal{B}^ \top = \begin{bmatrix} {{f}}_\mathcal{B} ^ \top & {\mu}_\mathcal{B}^ \top \end{bmatrix}$}$ is the wrench acting on a point of a rigid body expressed in the frame $\mathcal{B}$.
\item Given $\prescript{}{}{\mathrm{f}}_\mathcal{B}$ and $\prescript{}{}{\mathrm{f}}_\mathcal{C}$, $\prescript{}{}{\mathrm{f}}_\mathcal{C} = \prescript{\mathcal{C}}{}{X}_\mathcal{B} \prescript{}{}{\mathrm{f}}_\mathcal{B}$, where $\prescript{\mathcal{C}}{}{X}_\mathcal{B} \in \mathbb{R}^{6 \times 6}$ is the adjoint matrix defined as 
\scalebox{0.8}{$\prescript{\mathcal{C}}{}{X}_\mathcal{B} = \small{\begin{bmatrix}
\prescript{\mathcal{C}}{}{R}_\mathcal{B} & \prescript{\mathcal{C}}{}{p}_\mathcal{B}^{\wedge} \prescript{\mathcal{C}}{}{R}_\mathcal{B} \\
0_{3 \times 3} & \prescript{\mathcal{C}}{}{R}_\mathcal{B}
\end{bmatrix}}$}.
\end{itemize}
\looseness=-1

\subsection{Humanoid Robot Model}
A humanoid robot is a floating base multibody system composed of $n+1$ links connected by $n$ joints with one degree of freedom each. The joint positions $s$ and the homogeneous transformation from the inertial frame to the robot base frame $\mathcal{B}$ define the robot configuration. The configuration is identified by the triplet $q = (\prescript{\mathcal{I}}{}{p}_\mathcal{B}, \prescript{\mathcal{I}}{}{R}_\mathcal{B}, s) \in  \mathbb{R}^3 \times SO(3) \times \mathbb{R}^n$. The velocity of the system is characterized by the set $\nu = (\prescript{\mathcal{I}}{}{\dot{p}}_\mathcal{B}, \prescript{\mathcal{I}}{}{\omega}_\mathcal{B}, \dot{s})$ where $\prescript{\mathcal{I}}{}{\dot{p}}_\mathcal{B}$ is the linear velocity of the base frame expressed into the inertial frame, $\prescript{\mathcal{I}}{}{\omega}_\mathcal{B}$ is the angular velocity of the base frame expressed into the inertial frame, and $\dot{s}$ is the time derivative of the joint positions.
\looseness=-1

The equation of motion of a multibody system is described by applying the \textit{Euler-Poincar\'e Formalism}
\begin{equation}
M(q) \dot \nu + h(q, \nu) = B \tau + \sum_{k} {J_k(q)^\top f_{ext,k}} \; ,
\label{eq:robotdynamics}
\end{equation}
where \mbox{$M \in \mathbb{R}^{(6+n) \times (6+n)}$} is the mass matrix, \mbox{$h(q, \nu) \in \mathbb{R}^{6+n}$} accounts for the Coriolis, centrifugal and gravitational effects, $\tau \in \mathbb{R}^{n}$ are the joint torques, $B \in \mathbb{R}^{(6+n) \times n}$ is the selection matrix, $f_{ext,k} \in \mathbb{R}^6$ is the $k$-th contact wrench expressed in the contact frame, $J_k(q)$ is the Jacobian associated to the $k$-th contact wrench~\cite{featherstone2014rigid}.
Humanoid robots actuated by brushless direct current (DC) motors are equipped with high-ratio gearboxes to generate the substantial torques required to move the robot joints, given the robot weight. The dynamics of a DC motor with a harmonic drive is described by the equation $k_{t} i_m = J_m \ddot{\theta} + \frac{1}{r} \tau_F + \frac{1}{r} \tau$
where $k_t$ is the torque constant, $i_m$ is the applied motor current, $J_m \ddot{\theta}$ is the angular acceleration of the motor shaft, $\tau_F$ is the friction torque, $\tau$ is the torque applied to the load, and $r$ represents the reduction ratio~\cite{taghirad1996experimental}. In scenarios with high friction, high reduction ratios, low motor inertia, or low angular acceleration, the frictional effects dominate the system dynamics over the inertial effects of the motor, and $J_m \ddot{\theta} \ll \frac{1}{r} \tau_F$. The joint torque can be simplified in
\begin{equation}
\label{eq:HDmodel}
\tau = r k_t i_m -  \tau_F \, .
\end{equation}
\looseness=-1

\subsection{Mechanical friction torque model}
\label{sec:cv_scv}
The friction torque $\tau_F$ characterizes the nonlinear friction behavior of the harmonic reducer, encompassing static friction, dynamic friction, break-away force, pre-sliding displacement, frictional lag or hysteresis, and stick-slip~\cite{marques2016survey}. Static friction occurs during the sticking phase when the sliding velocity is zero, while dynamic friction is predominant during the sliding phase. Break-away force refers to the force required to overcome stiction. The Stribeck effect describes the decrease or negatively sloped characteristic of friction force at low sliding velocities, transitioning from static to Coulomb friction.

The physics-based friction models are classified into two categories: static models such as Coulomb, Coulomb-viscous, and Stribeck, and dynamic models such as Dahl, LuGre, and generalized Maxwell-slip~\cite{olsson1998friction}. Detailed mathematical formulations of dynamic models are beyond the scope of this work.
\looseness=-1

The continuous Coulomb-viscous (CV) model is defined as
\begin{equation}
\label{eq:cv}
    \tau_F = k_c \tanh(k_a \dot{s}) + k_v \dot{s} \; ,
\end{equation}
where $k_c \tanh(k_a \dot{s})$ represents the smooth Coulomb friction~\cite{pennestri2016review}, $k_v \dot{s}$ represents viscous friction. $\tanh$ approximates the sign function without discontinuity, and $k_a$ determines determining how quickly the friction changes as the velocity changes.
The Stribeck-Coulomb-viscous (SCV) model integrates the Stribeck effect in the previous formulation~\cite{olsson1998friction} and is defined as
\begin{equation}
\label{eq:scv}
    \tau_F = k_v \dot{s} + k_c \tanh(k_a \dot{s}) + \left( k_s {-} k_c  \right) e^{-\abs{\frac{\dot{s}}{v_s}}^\alpha} \tanh(k_a \dot{s}) \; ,
\end{equation}
where $v_s$ is the Stribeck velocity, $k_s$ represents the breakaway friction, and $\alpha$ is an empirical parameter determining how fast the static friction component fades away as the velocity increases.
\looseness=-1

\subsection{Physics informed neural networks}
Physics-Informed Neural Networks (PINNs) are a specialized class of neural networks that incorporate physical laws directly into the learning process, providing a robust alternative to traditional data-driven models~\cite{abiodun2018state,cuomo2022scientific}. Unlike conventional NNs, which rely solely on observational data, PINNs leverage governing physics equations to guide training, allowing the model to infer accurate solutions even from sparse data. This is accomplished by embedding physical laws as constraints within the model, ensuring that the solutions adhere to established scientific principles. The training of a PINN involves minimizing a composite loss function that combines both data-driven and physics-based terms. Given a general differential equation $\mathcal{N}(\tau_F) = 0$, where $\mathcal{N}$ represents a differential operator and $\tau_F$ is the unknown solution, the total loss function $\mathcal{L}$ is formulated as
\begin{IEEEeqnarray}{cl}
    \IEEEnonumber
\mathcal{L} = \mathcal{L}_{\text{data}} + \mathcal{L}_{\text{physics}} &= \lambda_{\text{data}} \frac{1}{N} \sum_{i=1}^{N} \left( {\tau}_{F,\text{pred}}(\mathbf{x}_i) - {\tau}_{F,\text{true}}(\mathbf{x}_i) \right)^2 \\ &+ \lambda_{\text{physics}} \frac{1}{M} \sum_{j=1}^{M} \left( \mathcal{N}({\tau_F}_{\text{pred}}(\mathbf{x}_j)) \right)^2,
\label{eq:loss}
\end{IEEEeqnarray}
where $\mathcal{L}_{\text{data}}$ minimizes the difference between predicted and true values at data points $\mathbf{x}_i$, and $\mathcal{L}_{\text{physics}}$ ensures that the predicted solution ${\tau}_{F,\text{pred}}$ satisfies the governing physical laws at collocation points $\mathbf{x}_j$. Here, $N$ is the number of available data points, $M$ is the number of collocation points where the differential equation is evaluated, and $\lambda_{\text{data}},\lambda_{\text{physics}}$ are scaling parameters that balance the two loss components. PINNs have proven to be powerful tools across numerous domains, solving complex physical problems where traditional models often struggle due to insufficient data or the need to incorporate domain-specific knowledge.
\looseness=-1
