\section{Conclusions}
\label{sec:conclusions}

The application of Physics-Informed Neural Networks (PINNs) for friction identification in humanoid robots offers a promising method, though there are considerations. Training directly on the robot may yield less accurate data than dedicated setups. However, utilizing existing sensor data and physical laws eliminates the need for additional hardware, which is advantageous. In addition, comprehensive validation across all joints is essential due to their differing friction characteristics, requiring specific hyperparameter tuning for each joint. PINNs present trade-offs in data needs, interpretability, and accuracy compared to traditional friction models, which can complicate their adoption in certain scenarios. Nonetheless, PINNs excel in capturing non-linear and dynamic friction effects that conventional models struggle with. Their scalability across multiple joints without rigid-body assumptions enhances their applicability in dynamic environments, facilitating more robust friction compensation and motion control in humanoid robots. Thus, while acknowledging these limitations, the benefits of employing PINNs underscore their potential to advance friction estimation capabilities in robotic systems.\looseness=-1

Future work should prioritize validating all joints of the ergoCub robot to ensure robustness across varying friction characteristics and refining joint-specific hyperparameter tuning. Additionally, we aim to integrate PINN models into joint torque control architectures for complex real-time applications, such as demonstrating their performance during the robot's walking scenarios.
\looseness=-1
